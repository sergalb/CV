% basic part begin
\documentclass[12pt, a4paper]{report}
\usepackage[utf8]{inputenc}
\usepackage[russian]{babel}
\usepackage[OT1]{fontenc}
\usepackage{amsmath}
\usepackage{amsfonts}
\usepackage{amssymb}
\usepackage[misc]{ifsym}
\usepackage{makeidx}
\usepackage{wasysym}
\usepackage{enumitem}
% basic part end

\usepackage{multicol}
\usepackage{fontawesome}
\usepackage{hyperref}
\usepackage[left=1cm,right=1cm,top=1cm,bottom=1cm]{geometry}

\author{Балахнин Сергей}

\setlength\columnsep{15mm}
\setlength\parindent{0pt}
\pagenumbering{gobble}

\begin{document}

    \section*{Балахнин Сергей}

    \par\hbox{\large\textbf{Образование}}\kern5pt\hrule\kern5pt

    \textbf{Студент 4 курса КТ ИТМО.}
    \hfill
    \textbf{ Средний балл: 4 / 5} \\
    Специализация: прикладная математика и информатика (01.03.02) \\
    Закончил 4 полные курса обучения в 2021 году. Но ушел в академ по состоянию здоровья.\\

    \par\hbox{\large\textbf{Релевантные к теме диплома курсы}}\kern5pt\hrule\kern5pt
    \begin{itemize}
        \item Алгоритмы и структуры данных \hfill 4 семестра на 5
        \item Распределенные системы \hfill 5
        \item Параллельное программирование \hfill 4
    \end{itemize}

    \par\hbox{\large\textbf{Опыт работы}}\kern3pt\hrule\kern10pt

    Backand-developer JetBrains, Qodana (статический анализатор) \\
    Стек: Java, Kotlin \\
    \underline{Период: 09.2021 - настоящее время, 35 часов в неделю} \\

    Стажировка JetBrains, Qodana \\
    Проект по генерации стабильных идентификаторов элементов AST для нахождения тех же элементов при изменении кода \href{https://internship.jetbrains.com/projects/871/}{описание проекта} \\
    Результат — увеличил точность нахождения элементов на 20\% для Java \\
    \underline{Период: 07.2021 - 09.2021} \\

    Стажировка Huawei, LLVM \\
    Работал над Profile Guided Optimization \\
    \underline{Период: 07.2020 - 02.2021} \\

    Стажировка в Тинькофф, проект kassir.ru. \\
    Стек: Java, Kotlin, Spring \\
    \underline{Период: 07.2019 - 09.2019} \\


    \par\hbox{\large\textbf{Ключевые навыки}}\kern5pt\hrule\kern5pt

    \begin{itemize}

        \item \textbf{Языки программирования.}

        Java, Kotlin. В меньшей мере Haskell, Python, C++.

        \item Знание алгоритмов, структур данныx
        \item Знание теоретических основ многопоточности, практический опыт использования
        \item Базовые знания в машинном обучении
        \item Intermediate английский язык
    \end{itemize}

    \par\hbox{\large\textbf{Учебные проекты, достижения}}\kern5pt\hrule\kern5pt
    \begin{itemize}

        \item GitHub-репозиторий с ссылками на все учебные проекты \textbf{\url{https://github.com/sergalb/itmo}}

        \item Генератор парсеров LALR грамматик (ослабленный аналог antlr)

        Стек: Kotlin, antlr\\
        Ссылка на проект:
        \textbf{
        \url{https://github.com/sergalb/ParserGenerator}
        }

        \item Утилита для поиски данной строки в файлах (аналог grep, с UI)

        Стек: C++, QtCreator, Multithreading.\\
        В этом проекте удалось добиться скорости работы, превосходящей grep (за счет предподсчета)\\
        Сылка на проект:
        \textbf{
        \url{https://github.com/sergalb/cpp-third-term/tree/master/SubstringFinder}
        }

        \item Учебные android-приложения

        Стек: Kotlin, Multithreading, REST\\
        Сылка на проект:
        \textbf{
        \url{https://github.com/sergalb/android-2019}
        }

        \item Победы и призерства в школьных математических олимпиадах - "Росатом", "Физтех", "ОММО", "Будущие исследователи - будущее науки"

    \end{itemize}

\end{document}