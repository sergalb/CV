% basic part begin
\documentclass[12pt, a4paper]{report}
\usepackage[utf8]{inputenc}
\usepackage[russian]{babel}
\usepackage[OT1]{fontenc}
\usepackage{amsmath}
\usepackage{amsfonts}
\usepackage{amssymb}
\usepackage[misc]{ifsym}
\usepackage{makeidx}
\usepackage{wasysym}
\usepackage{enumitem}
% basic part end

\usepackage{multicol}
\usepackage{fontawesome}
\usepackage{hyperref}
\usepackage[left=1cm,right=1cm,top=1cm,bottom=1cm]{geometry}

\author{Балахнин Сергей}

\setlength\columnsep{15mm}
\setlength\parindent{0pt}
\pagenumbering{gobble}

\begin{document}

    \section*{Балахнин Сергей}

    {\large\textbf{Backend-разработчик}}

    \hbox{\large\textbf{Контакты}}\kern5pt\hrule\kern5pt

    \faPhone: +7-952-265-18-42

    \Letter: sergey-dzr2@yandex.ru

    \faGithub: \url{http://www.github.com/sergalb}

    \faSend: \url{http://www.t.me/serg_alb}

    \faVk: \url{http://www.vk.com/serg_alb} \\



    \par\hbox{\large\textbf{Образование}}\kern5pt\hrule\kern5pt

    \textbf{Студент 3 курса КТ ИТМО.}
    \hfill
    \textbf{ Средний балл: 4 / 5} \\
    Специализация: прикладная математика и информатика (01.03.02), проходной балл в год поступления - 309/300.\\

    \par\hbox{\large\textbf{Дополнительные курсы}}\kern5pt\hrule\kern5pt
    Совместный курс по машинному обучению от Huawei и лаборатории ИТМО \\
    \underline{Период прохождения: 10.2019 - настоящее время} \\


    \par\hbox{\large\textbf{Опыт работы}}\kern3pt\hrule\kern10pt

    Стажировка в Тинькофф, проект kassir.ru. \\
    Стек: Java, Kotlin, Spring \\
    \underline{Период: 07.2019 - 09.2019} \\


    \par\hbox{\large\textbf{Ключевые навыки}}\kern5pt\hrule\kern5pt

    \begin{itemize}

        \item \textbf{Языки программирования.}

        Java, Kotlin. В меньшей мере Haskell, python, C++.

        \item Опыт работы с git
        \item Знание алгоритмов, структур данныx
        \item Знание теоретических основ многопоточности, практический опыт использования
        \item Базовые знания в машинном обучении
        \item web-программирование, spring
        \item Программирование под android
        \item Английский язык на уровне чтения документации
        \item Умею разбираться в больших проектах\\
    \end{itemize}

    \par\hbox{\large\textbf{Учебные проекты, достижения}}\kern5pt\hrule\kern5pt
    \begin{itemize}

        \item GitHub-репозиторий с ссылками на все учебные проекты \textbf{\url{https://github.com/sergalb/itmo}}

        \item Генератор парсеров LALR грамматик (ослабленный аналог antlr)

        Стек: Kotlin, antlr\\
        Ссылка на проект:
        \textbf{
        \url{https://github.com/sergalb/ParserGenerator}
        }

        \item Утилита для поиски данной строки в файлах (аналог grep, с UI)

        Стек: C++, QtCreator, Multithreading.\\
        В этом проекте удалось добиться скорости работы, превосходящей grep (за счет предподсчета)\\
        Сылка на проект:
        \textbf{
        \url{https://github.com/sergalb/cpp-third-term/tree/master/SubstringFinder}
        }

        \item Учебные android-приложения

        Стек: Kotlin, Multithreading, REST\\
        Сылка на проект:
        \textbf{
        \url{https://github.com/sergalb/android-2019}
        }

        \item Победы и призерства в школьных математических олимпиадах - "Росатом", "Физтех", "ОММО", "Будущие исследователи - будущее науки"

    \end{itemize}

\end{document}