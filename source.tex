% basic part begin
\documentclass[12pt, a4paper]{report}
\usepackage[utf8]{inputenc}
\usepackage[russian]{babel}
\usepackage[OT1]{fontenc}
\usepackage{amsmath}
\usepackage{amsfonts}
\usepackage{amssymb}
\usepackage[misc]{ifsym}
\usepackage{makeidx}
\usepackage{wasysym}
\usepackage{enumitem}
% basic part end

\usepackage{multicol}
\usepackage{fontawesome}
\usepackage{hyperref}
\usepackage[left=1cm,right=1cm,top=1cm,bottom=1cm]{geometry}

\author{Балахнин Сергей}

\setlength\columnsep{15mm}
\setlength\parindent{0pt}
\pagenumbering{gobble}

\begin{document}

    \section*{Балахнин Сергей}

    {\large\textbf{Backend-разработчик}}

    \hbox{\large\textbf{Контакты}}\kern5pt\hrule\kern5pt

    \faPhone: +7-952-265-18-42

    \Letter: serg17alb@gmail.com

    \faGithub: \url{http://www.github.com/sergalb}

    \faSend: \url{http://www.t.me/serg_alb}

    \faLinkedinSquare: \url{https://www.linkedin.com/in/sergey-balakhnin-790180228/}\\


    \par\hbox{\large\textbf{Опыт работы}}\kern3pt\hrule\kern10pt
    JetBrains, Qodana team. \\
    Стек: Kotlin, Java, Static Analysis \\
    Специализация: Разработка в core команде статического анализатора.\\
    \underline{Period: 7.2021 - 03.2022} \\

    Huawei, Programming Language team. \\
    Стек: C/C++, research \\
    Специализация: Разработка транслятора из Java в другой язык программирования\\
    \underline{Period: 10.2020 - 04.2021} \\

    Huawei, project - LLVM compiler. \\
    Stack: C/C++, research \\
    Специализация: Разработка Profile guided optimization для clang.\\
    \underline{Period: 06.2020 - 10.2020} \\

    Стажировка в Тинькофф, проект kassir.ru. \\
    Стек: Java, Kotlin, Spring \\
    \underline{Период: 06.2019 - 09.2019} \\

    \par\hbox{\large\textbf{Образование}}\kern5pt\hrule\kern5pt

    \textbf{Бакалавр КТ ИТМО.}
    \hfill
    \textbf{ Средний балл: 4.1 / 5} \\
    Специализация: Прикладная математика и информатика.\\


    \par\hbox{\large\textbf{Ключевые навыки}}\kern5pt\hrule\kern5pt

    \begin{itemize}

        \item \textbf{Языки программирования.}

        Индустриальный опыт: Kotlin, Java. Учебный опыт: Haskell, Python, C++, Go.

        \item Знание алгоритмов, структур данныx
        \item Теоретический и практический опыт в многопоточном программировании.
        \item B1-B2 Английский язык
        \item Опыт работы с git
    \end{itemize}

    \par\hbox{\large\textbf{Учебные проекты, достижения}}\kern5pt\hrule\kern5pt
    \begin{itemize}

        \item \textbf{Дипломная работа}. Работал над проблемой поиска активного модуля. А именно рассширил подход описанный в статье: \href{https://bmcbioinformatics.biomedcentral.com/articles/10.1186/s12859-020-03572-9}{\textit{Markov chain Monte Carlo for active module identification problem}}\\
        В моей работе был поддержан поиск активного модуля для метаболических сетей (в оригинальной работе поддерживались только белок-белковые сети).\\
        Стек: R, C++
        Ссылка на проект:
        \textbf{
            \url{https://github.com/sergalb/mcmcRanking}
        }

        \item 3 года был волонтерос на NERC ICPC финале.

        \item Генератор парсеров LALR грамматик (ослабленный аналог antlr)

        Стек: Kotlin, antlr\\
        Ссылка на проект:
        \textbf{
        \url{https://github.com/sergalb/ParserGenerator}
        }

        \item Утилита для поиски данной строки в файлах (аналог grep, с UI)

        Стек: C++, QtCreator, Multithreading.\\
        В этом проекте удалось добиться скорости работы, превосходящей grep (за счет предподсчета)\\
        Сылка на проект:
        \textbf{
        \url{https://github.com/sergalb/cpp-third-term/tree/master/SubstringFinder}
        }

        \item Учебные android-приложения

        Стек: Kotlin, Multithreading, REST\\
        Сылка на проект:
        \textbf{
        \url{https://github.com/sergalb/android-2019}
        }

        \item Победы и призерства в школьных математических олимпиадах - "Росатом", "Физтех", "ОММО", "Будущие исследователи - будущее науки"

    \end{itemize}

\end{document}